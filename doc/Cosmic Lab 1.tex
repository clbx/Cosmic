%%%%%%%%%%%%%%%%%%%%%%%%%%%%%%%%%%%%%%%%%
% fphw Assignment
% LaTeX Template
% Version 1.0 (27/04/2019)
%
% This template originates from:
% https://www.LaTeXTemplates.com
%
% Authors:
% Class by Felipe Portales-Oliva (f.portales.oliva@gmail.com) with template 
% content and modifications by Vel (vel@LaTeXTemplates.com)
%
% Template (this file) License:
% CC BY-NC-SA 3.0 (http://creativecommons.org/licenses/by-nc-sa/3.0/)
%
%%%%%%%%%%%%%%%%%%%%%%%%%%%%%%%%%%%%%%%%%

%----------------------------------------------------------------------------------------
%	PACKAGES AND OTHER DOCUMENT CONFIGURATIONS
%----------------------------------------------------------------------------------------

\documentclass[
	12pt, % Default font size, values between 10pt-12pt are allowed
	%letterpaper, % Uncomment for US letter paper size
	%spanish, % Uncomment for Spanish
]{fphw}

% Template-specific packages
\usepackage[utf8]{inputenc} % Required for inputting international characters
\usepackage[T1]{fontenc} % Output font encoding for international characters
\usepackage{mathpazo} % Use the Palatino font

\usepackage{graphicx} % Required for including images

\usepackage{booktabs} % Required for better horizontal rules in tables

\usepackage{listings} % Required for insertion of code

\usepackage{enumerate} % To modify the enumerate environment

\usepackage{hyperref}
\hypersetup{
    colorlinks=true,
    linkcolor=blue,
    filecolor=magenta,      
    urlcolor=cyan,
}

%----------------------------------------------------------------------------------------
%	ASSIGNMENT INFORMATION
%----------------------------------------------------------------------------------------

\title{Assembly Lab \#1: Basic Operations} % Assignment title


\date{March 28th, 2025} % Due date

\institute{Elizabethtown College \\ Department of Computer Science} % Institute or school name

\class{Digital Design II} % Course or class name

\professor{Professor X} % Professor or teacher in charge of the assignment

%----------------------------------------------------------------------------------------

\begin{document}

\maketitle % Output the assignment title, created automatically using the information in the custom commands above

%----------------------------------------------------------------------------------------
%	ASSIGNMENT CONTENT
%----------------------------------------------------------------------------------------

\section*{Lab Objective}

\begin{problem}
	To obtain the basics of assembly such as moving data around, doing basic arithmetic, and jumps. 
\end{problem}

%------------------------------------------------

\subsection*{Prelab}

\begin{itemize}
  \item Set up the Cosmic environment and assembler. (Make sure you have Python 3.x installed)
  \item Familiarize yourself with the documentation of the opcodes, labels, and variables.
  \item In particular, pay attention to the following opcodes: MOV, ADD, INC, JMP, CMP.
\end{itemize}

%----------------------------------------------------------------------------------------

\section*{During Lab}

\begin{problem}
	You will write a simple program that will take a number from a position in memory, calculate the correlating Fibonacci number, and put it in another position of memory. 
\end{problem}

%------------------------------------------------

\begin{itemize}
  \item Store a number, 1 $\leq$ n $\leq$ 13, in an arbitrary position of general memory (you may use a variable for this).
  \item Based on that number, calculate the correlating nth Fibonacci number.
  \item You will do this with a loop, you must use labels and the JMP instruction for this.
  \item Once the number is calculated, store it elsewhere in memory (you may use a variable for this)
\end{itemize}

%----------------------------------------------------------------------------------------

\section*{Grading}

30\% - Program assembles\\
40\% - Program calculates Fibonacci number\\
10\% - Proper memory space is used\\
10\% - Comments added\\
10\%  - Loop runs in less than 30 cycles\\


%------------------------------------------------

\section*{Helpful Links}
\href{https://www.mathsisfun.com/numbers/fibonacci-sequence.html}{Fibonacci Numbers}\\
\href{https://github.com/clbx/Cosmic/tree/master/doc}{Cosmic Documentation}\\

%----------------------------------------------------------------------------------------

\end{document}
